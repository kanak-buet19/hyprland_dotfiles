%% 
%% Copyright 2007-2025 Elsevier Ltd
%% 
%% This file is part of the 'Elsarticle Bundle'.
%% ---------------------------------------------
%% 
%% It may be distributed under the conditions of the LaTeX Project Public
%% License, either version 1.3 of this license or (at your option) any
%% later version.  The latest version of this license is in
%%    http://www.latex-project.org/lppl.txt
%% and version 1.3 or later is part of all distributions of LaTeX
%% version 1999/12/01 or later.
%% 
%% The list of all files belonging to the 'Elsarticle Bundle' is
%% given in the file `manifest.txt'.
%% 
%% Template article for Elsevier's document class `elsarticle'
%% with numbered style bibliographic references
%% SP 2008/03/01
%% $Id: elsarticle-template-num.tex 272 2025-01-09 17:36:26Z rishi $
%%
\documentclass[preprint,12pt]{elsarticle}

%% Use the option review to obtain double line spacing
%% \documentclass[authoryear,preprint,review,12pt]{elsarticle}

%% Use the options 1p,twocolumn; 3p; 3p,twocolumn; 5p; or 5p,twocolumn
%% for a journal layout:
%% \documentclass[final,1p,times]{elsarticle}
%% \documentclass[final,1p,times,twocolumn]{elsarticle}
%% \documentclass[final,3p,times]{elsarticle}
% \documentclass[final,3p,times,twocolumn]{elsarticle}
%% \documentclass[final,5p,times]{elsarticle}
% \documentclass[final,5p,times,twocolumn]{elsarticle}

%% For including figures, graphicx.sty has been loaded in
%% elsarticle.cls. If you prefer to use the old commands
%% please give \usepackage{epsfig}

%% The amssymb package provides various useful mathematical symbols
\usepackage{amssymb}
%% The amsmath package provides various useful equation environments.
\usepackage{amsmath}
\usepackage{times}  
%% The amsthm package provides extended theorem environments
%% \usepackage{amsthm}

%% The lineno packages adds line numbers. Start line numbering with
%% \begin{linenumbers}, end it with \end{linenumbers}. Or switch it on
%% for the whole article with \linenumbers.
%% \usepackage{lineno}

\usepackage{balance}
\usepackage{array}
\usepackage[english]{babel}
% \usepackage[format=plain,justification=justified,singlelinecheck=false,font={stretch=1.125,small,sf},labelfont=bf,labelsep=space]{caption}

\usepackage[format=plain,justification=justified,singlelinecheck=false,
            font={stretch=1.125,small},labelfont=bf,labelsep=space]{caption}

\usepackage{charter}
\usepackage{colortbl}
\usepackage{color,soul}
\usepackage{droidsans}
\usepackage{epstopdf}%This line makes .eps figures into .pdf - please comment out if not required.
\usepackage{extsizes}
\usepackage{fancyhdr}
\usepackage{float}
\usepackage{fnpos}
\usepackage[T1]{fontenc}
\usepackage[left=1.5cm, right=1.5cm, top=1.785cm, bottom=2.0cm]{geometry}
\usepackage{graphicx} 
\usepackage[colorlinks=true, citecolor=blue]{hyperref}
\usepackage{lastpage}
\usepackage{makecell}
\usepackage{multirow}
\usepackage{mathptmx}
\usepackage[numbers]{natbib}
\usepackage[abs]{overpic} 
\usepackage{sectsty}
\usepackage{setspace}
\usepackage{subcaption}
\usepackage{tabularx}
\usepackage[compact]{titlesec}
\usepackage[usenames,dvipsnames]{xcolor}

\usepackage[textwidth=1.5cm]{todonotes}
\usepackage{soul}

\usepackage{booktabs} 
\usepackage{adjustbox}
\doublespacing


\sethlcolor{yellow}

% Let soul handle these commands/macros inside \hl
\soulregister\ref7
\soulregister\pageref7
\soulregister\NbC7
\soulregister\NbCO7

% Reviewer changes toggle
\newif\ifshowrevisions
\showrevisionstrue        % <-- change to \showrevisionsfalse for clean version
\newcommand{\rev}[1]{\ifshowrevisions\hl{#1}\else#1\fi}


\newcommand{\NbC}[1]{Nb\textsubscript{#1}C}
\newcommand{\NbCO}[1]{Nb\textsubscript{#1}CO\textsubscript{#1}}
\newcommand{\NbCLi}[1]{Nb\textsubscript{#1}CLi}
\newcommand{\NbCNa}[1]{Nb\textsubscript{#1}CNa}


\journal{Journal of Physics and Chemistry of Solids}

\begin{document}

\begin{frontmatter}

%% Title, authors and addresses

%% use the tnoteref command within \title for footnotes;
%% use the tnotetext command for theassociated footnote;
%% use the fnref command within \author or \affiliation for footnotes;
%% use the fntext command for theassociated footnote;
%% use the corref command within \author for corresponding author footnotes;
%% use the cortext command for theassociated footnote;
%% use the ead command for the email address,
%% and the form \ead[url] for the home page:
%% \title{Title\tnoteref{label1}}
%% \tnotetext[label1]{}
%% \author{Name\corref{cor1}\fnref{label2}}
%% \ead{email address}
%% \ead[url]{home page}
%% \fntext[label2]{}
%% \cortext[cor1]{}
%% \affiliation{organization={},
%%             addressline={},
%%             city={},
%%             postcode={},
%%             state={},
%%             country={}}
%% \fntext[label3]{}

\title{High-Fidelity Multiphysics Simulation of Melt Pool and
Keyhole Dynamics in Laser Powder Bed Fusion: Insights from NIST AM-Bench 2025}

\author[BUET]{Badhon Kumar\corref{cor1}}
\author[BUET]{Rakibul Islam Kanak}
\author[NU]{Jiachen Guo}
\author[NU]{Wing Kam Liu}

\author[UD]{Abdullah Al Amin\corref{cor2}}

\cortext[cor1]{%
  Corresponding author.  
  Department of Mechanical Engineering, Bangladesh University of Engineering and Technology, Dhaka 1000, Bangladesh.
  Email address: badhan.me19@gmail.com%
}

\cortext[cor2]{%
  Corresponding author.  
  Department of Mechanical and Aerospace Engineering, University of Dayton, 300 College Park Drive, Dayton, OH 45459, USA.
  Email address: aamin1@udayton.edu%
}

\affiliation[BUET]{organization={Department of Mechanical Engineering, Bangladesh University of Engineering and Technology, Dhaka},%Department and Organization
            addressline={}, 
            city={Dhaka}, postcode={1000}, state={}, country={Bangladesh}}

\affiliation[UD]{organization={Department of Mechanical and Aerospace Engineering, University of Dayton},%Department and Organization
                        addressline={300 College Park Drive}, 
                        city={Dayton},
                        postcode={45469}, 
                        state={OH},
                        country={USA}}

\affiliation[NU]{organization={Department of Mechanical, Northwestern University}, addressline={2145 Sheridan Road, Room B224}, city={Evanston}, postcode={60208}, state={IL}, country={USA}}



%% Abstract
\begin{abstract}
%% Text of abstract
Metal Laser Powder Bed Fusion (PBF-LB/M) is a leading additive manufacturing technique, yet the process remains highly sensitive to laser-material interactions that govern melt pool dynamics and keyhole formation. In particular, the transition from conduction to keyhole mode is critical, as unstable keyholes can lead to porosity and reduced part integrity. To address this challenge, we discuss a high-fidelity multiphysics simulation framework based on the open source finite volume method (FVM) solver package laserbeamFoam for metal PBF-LB/M that captures coupled phenomena of heat transfer, fluid flow, recoil pressure, and vaporization. The model resolves melt pool evolution in three dimensions with sufficient temporal and spatial resolution to predict keyhole initiation, stability, and collapse. Simulations are performed across a range of laser powers and scan speeds, enabling direct comparison with benchmark experiments conducted at the National Institute of Standards and Technology (NIST) in 2022 and 2025. Quantitative validation against in-situ synchrotron X-ray imaging demonstrates excellent agreement in melt pool depth, width, and keyhole morphology. The model accurately predicts the onset of keyhole porosity under high energy densities, reproducing experimentally observed thresholds within experimental uncertainty. These results highlight the predictive capability of physics-based LPBF models, paving the way for process optimization, defect mitigation, and the integration of simulation into digital twin frameworks for additive manufacturing.
\end{abstract}



%%Graphical abstract
\begin{graphicalabstract}
      % \includegraphics[width=\columnwidth]{graphicalAbstract.pdf}
%\includegraphics{grabs}
\end{graphicalabstract}

%%Research highlights
\begin{highlights}
  \item High-fidelity finite volume simulations capture coupled heat transfer, fluid flow, recoil pressure, and vaporization in LPBF.
  \item Model resolves melt pool geometry and predicts keyhole initiation, stability, and collapse under varying process conditions.
  \item Quantitative validation against NIST synchrotron X-ray experiments shows excellent agreement in melt pool depth, width, and keyhole morphology.
  \item Simulations accurately reproduce experimentally observed thresholds for keyhole porosity formation at high energy densities.
  \item Demonstrates predictive capability for defect mitigation and establishes a pathway toward digital twin integration in additive manufacturing.

\end{highlights}

  

%% Keywords
\begin{keyword}
%% keywords here, in the form: keyword \sep keyword
MXene \sep Raman Spectroscopy \sep Density Functional Theory \sep Ion Diffusion \sep Lithium/Sodium-Ion Battery


%% PACS codes here, in the form: \PACS code \sep code

%% MSC codes here, in the form: \MSC code \sep code
%% or \MSC[2008] code \sep code (2000 is the default)

\end{keyword}

\end{frontmatter}


\input{sections/intro.tex}
\input{sections/methodology.tex}
\input{sections/validation.tex}
\input{sections/results.tex}
\input{sections/conslusion.tex}









\section*{Author contributions} 

\textbf{Badhon Kumar:} Conceptualization, Methodology, Formal Analysis, Writing-Original Draft, Writing-Review \& Editing. \textbf{Rakibul Islam:} Editing \& Manuscript Revision. \textbf{Jiachen Guo:} Editing \& Manuscript Revision. \textbf{Wing Kam Liu:} Editing \& Manuscript Revision. \textbf{Abdullah Amin:} Conceptualization, Writing-Original Draft, Resources, Manuscript Revision 

\section*{Conflicts of interest}
The authors declare no conflicts of interest.

\section*{Data availability}
Data is available upon reasonable request.

\section*{Acknowledgements}

The authors acknowledge gracious support from the Ohio Supercomputer Center for the high performance computation resources through academic project. Abdullah Al Amin acknowledges support from the School of Engineering at the University of Dayton through the Research Council Seed Grant.



%% Use \subsection commands to start a subsection.
% \section{Example Subsection}
% \label{introduction}

% Subsection text.

%% Use \subsubsection, \paragraph, \subparagraph commands to 
%% start 3rd, 4th and 5th level sections.
%% Refer following link for more details.
%% https://en.wikibooks.org/wiki/LaTeX/Document_Structure#Sectioning_commands


%% The Appendices part is started with the command \appendix;
%% appendix sections are then done as normal sections
\appendix
\label{supplementaryInformation}

\setcounter{figure}{0}
\renewcommand{\thefigure}{A\arabic{figure}}
\setcounter{table}{0}
\renewcommand{\thetable}{A\arabic{table}}
\section{Supplementary Information}




\clearpage

\bibliographystyle{elsarticle-num} % or another appropriate style
\bibliography{reference.bib}       % replace with your actual .bib filename without the extension

\end{document}

\endinput
%%
%% End of file `elsarticle-template-num.tex'.
