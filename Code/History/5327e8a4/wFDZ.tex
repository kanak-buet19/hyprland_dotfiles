\section{Introduction}
\label{sec:intro}





Despite d its recent origins in the mid-1990s, Laser Powder Bed Fusion (LPBF) has revolutionized the manufacturing sector by enabling production of complex and intricate geometries through freeform fabrication techniques, making it critical for industries such as aerospace and biomedical engineering \cite{herzogAdditiveManufacturingMetals2016,gorsseAdditiveManufacturingMetals2017,wangEffectsEnergyDensity2021}. This potential comes with significant process complexity that requires careful control to avoid manufacturing defects like porosity, balling, lack-of-fusion, stair-step effect, spattering, denduation and many more \cite{xuNumericalSimulationMelt2022a,nabaviComprehensiveReviewRecent2024,akterjahanMultiscaleModelingFramework2022a,weiMechanisticModelsAdditive2021} . Therefore, proper control, optimization and understanding of key process parameters used in this process such as, scan speed, layer thickness, laser power, spot radius, turn-around time are crucial. Computational modelling approaches can greatly complement the experimental techniques from both financial and logistical perspective. To make these models reliable, validation against proper experimental data is a must.



The National Institute of Standards and Technology (NIST) provides exactly the platform that not only enables modelers with experimental datasets generated using state-of-the-art experimental standards, but also gives modelers a comprehensive validation framework. The primary goal of these benchmarks is to allow researchers to test their predictive simulations against highly controlled experimental data, contributing to the broader effort of developing predictive modeling and simulation tools crucial for the qualification and certification of additively manufactured components.


\begin{figure}
    \centering
    \includegraphics[width=0.8\textwidth]{figures/schematic_2.pdf}
    \caption{Illustration of laser scan strategy for 5 mm x 5 mm pad with a 0.75 ms laser turnaround time}
    \label{fig:scan_strategy}
\end{figure}


This study reports on the NIST AM Bench 2025 Challenge 06 PMPG (Pad Melt Pool Geometry) and presents validated methods for quantitatively predicting the geometrical characteristics of multi-track LPBF processes. This challenge specifically focuses on the effect of varying powder layer thickness on the fabrication of arrays of adjacent laser tracks (pads). The experiments for this challenge were performed using the NIST Fundamentals of Laser-Material Interaction (FLaMI) metrology platform.

\begin{figure}
    \centering
    \includegraphics[width=\textwidth]{figures/schematic_1.pdf}
    \caption{Cross-sectional schematic of melt-pool measurands; (a) bead height, (b) depth and overlap depth, (c) width and overlap width, (d) solidified and dilution layer areas, and (e) representative melt-pool image of the final track. }
    \label{fig:schematic_meltpool}
\end{figure}



The AMB2025-06-PMPG challenge, explained in the NIST challenge 06 description document \cite{nist_amb2025_tech}, examines three distinct powder layer conditions using two pad geometries: $5 \,\text{mm} \times 5 \,\text{mm}$ and $1 \,\text{mm} \times 5 \,\text{mm}$. The experimental conditions comprise a bare substrate ($0 \,\mu\text{m}$ powder), an $80 \,\mu\text{m}$ powder layer, and a $160 \,\mu\text{m}$ powder layer. Process parameters remain constant across all tests: laser power of $285 \,\text{W}$, scanning speed of $960 \,\text{mm/s}$, hatch spacing of $0.11 \,\text{mm}$, and Gaussian beam spot size of $72 \,\mu\text{m}$. The challenge incorporates ex-situ characterization of track topography and cross-sectional geometry.

The core modeling tasks for AMB2025-06 are divided into two challenge problems. The first is the Pad Melt Pool Geometry (CHAL-AMB2025-06-PMPG), which requires modelers to calculate geometrical measurements for each pad cross-section, specifically the average bead height, depth, overlap depth, and width. Additionally, modelers must calculate the total solidified area above the substrate and the total dilution area below the substrate. The second problem is the Pad Surface Topography (CHAL-AMB2025-06-PST), which requires the calculation of the fused layer thickness and the root-mean-square-height ($S_q$). The required surface topography metrics must be calculated for various specified evaluation areas and profiles within the $5 \,\text{mm} \times 5 \,\text{mm}$ and $1 \,\text{mm} \times 5 \,\text{mm}$ pad geometries.

While previous numerical studies typically examined single tracks or, in limited cases, up to 8 tracks \cite{quevaMesoscaleMultilayerMultitrack2024}, this challenge required results averaged from 45-track simulations, thereby pushing the boundaries of current computational modeling capabilities. Interestingly, for the 3 cases (bareplate, $80\,\mu\text{m}$ and $160\,\mu\text{m}$), since the same processing parameters were used, this approach focuses to check the model's reliability in predicting results in a unified scenario. This is particularly important because, thermophysical properties are inherently different in powderbed cases compared to bareplate due to their porous structure. Addressing the effect of turn-around time is another unique addition. 

Different modelling techniques have been proposed in recent years that predict the meltpool morphology during the LPBF process. Among them, the simplest approach is using the thermal heat conduction equation. Although this method is less computationally intensive, it ignores the convection inside liquid metal resulting in an overestimation of peak-temperature and cooling rate \cite{debroyAdditiveManufacturingMetallic2018}. To account for fluid flow and meltpool convection, some researchers have used FDM (Finite Difference Method) based models that solve both thermal and Navier-Stokes equations \cite{manvatkarSpatialVariationMelt2015}. Amin et al.\cite{aminPhysicsGuidedHeat2024}  have used Finite Volume Method (FVM) based solver to predict the meltpool dimensions and cooling rate accurately. The main limitation of these solvers is that all of them consider meltpool top surface to be flat, but in practical scenario, the free surface is not flat. Due to the inherent assumption of these solvers, practically these methods cannot be used for powderbed based simulations because they will not be able to properly capture the resulting bead height. % we can also talk about this paper \cite{le2020_flatsurfacepowderbed} that performed LPBF on flat surface.

To track the free surface during the LPBF simulation, different techniques such as Volume of Fluid (VOF) \cite{hirtVolumeFluidVOF1981}, Level-set (LS) \cite{dervieuxFiniteElementMethod1980}, Lattice Boltzman method (LBM) \cite{chenLATTICEBOLTZMANNMETHOD1998} and Smoothed Particle Hydrodynamics (SPH) \cite{gingoldSmoothedParticleHydrodynamics1977} are used. Although these methods can potentially capture the sharp interface during the simulation, they are very computationally intensive. In recent years, some VOF solvers use Ray-tracing algorithm to better predict the resulting keyhole during the LPBF process \cite{renHighfidelityModellingSelective2021a,leStudyKeyholemodeMelting2019}. 

In this work, we employ the open-source LPBF solver \textit{LaserbeamFOAM}~\cite{flintLaserbeamFoamLaserRaytracing2023}, which is based on a finite-volume method (FVM) and incorporates a ray-tracing algorithm to capture multiple reflections during the LPBF process. To generate powder-bed particles, we use the open-source DEM solver \textit{LIGGGHTS}~\cite{LIGGGHTSOpenSource}. The focus of this study is to investigate how an approach designed to mitigate abrupt, spurious particle velocities in the powder bed during the LPBF process may lead to discrepancies in melt-pool morphology when compared with experimental observations. Furthermore, we examine how the proposed modeling framework can consistently predict melt-pool morphologies under varying powder-layer thicknesses within a unified environment. Finally, both approaches are benchmarked against experimental data from the NIST AM-Bench 2025.


